\chapter{Increased monitoring by TEQSA}\label{chap:7}

The Tertiary Education Quality and Standards Agency (TEQSA) is the primary regulator of Australia's higher education system. It annually reviews university performance, including examining risk indicators such as attrition and fail rates.

But university-level statistics can hide poor results for small groups of students behind good results for most students. If TEQSA monitored attrition rates for part-time and low-ATAR students, and looked at what proportion of students fail multiple subjects, it could more easily identify problems with university performance

\section{Interpretation and enforcement of the threshold standards}\label{sec:7.1}

As noted in previous chapters, TEQSA enforces the `threshold standards' that all higher education providers must meet. The admission standards set out broad requirements for selecting students. TEQSA decides whether a provider's admission policies and practices comply with these standards, through annual monitoring and a re-registration process that occurs every seven years. TEQSA's signals to the higher education sector on how they will enforce the standards are therefore a significant influence on provider behaviour.

TEQSA interprets the admissions standards as primarily concerning academic ability rather than other obstacles to success, such as part-time study. On its `our commentary' webpage, designed to help higher education providers implement the threshold standards, TEQSA says it requires that admitted students must be `\ldots{}equipped to succeed'. Sample measures relate to academic preparation. Where it acknowledges potential cohorts `initially at some known risk of not succeeding', it uses as an example `educationally disadvantaged' students, rather than time-poor students, and its suggested response is `targeted support', rather than any change to the admission process itself (these, of course, are not mutually exclusive).\footcite{TEQSA2017}

Similarly, part-time study is not evidently recognised as a risk factor in the most detailed public evidence on TEQSA's enforcement of the new standards. TEQSA's approach is reflected by the conditions it imposes on courses taught by non-university higher education providers (their students are not included in this report's statistical analysis).\footnote{See also the general information on admissions required at: \textcite[][6--7]{TEQSA2017b}. Public universities accredit their own courses, so none of them have conditions attached to courses.} 
These conditions show TEQSA is concerned about English-language requirements, how prior learning is assessed, monitoring of student outcomes, benchmarking of those outcomes, high attrition of online students, and counselling students at risk of not completing.\footnote{See the conditions attached to course accreditations for the Australian College of Physical Education, the Australian Institute of Professional Counsellors, Melbourne Polytechnic, and Photography Holdings: \textcite[][]{TEQSA2017c}.} Part-time study may contribute to these problems, but it is not separately identified.

Although TEQSA has not focused regulatory conditions on part-time study, it discussed the topic in a report. In 2017, TEQSA released a statistical analysis of institution-level first-year attrition. It identified high rates of part-time enrolment as a risk factor in some non-university higher education providers. For universities, however, it did not identify part-time study as a risk factor.\footcite[][19--23]{TEQSA2016e} 
TEQSA's current approach seems to ignore the risk to completion associated with part-time study.

TEQSA could do more within its current legal powers to regulate the way universities handle part-time students. It could rely on the legal requirement that universities admit students only if there are `\ldots{}no known limitations that would be expected to impede {[}the admitted student's{]} progression and completion'.\footcite[][3]{DepartmentofEducationandTraining2015n}
Part-time study is a `known limitation'. It is a greater completion risk than `educational disadvantage' (which is regulated), because part-time study can impede students who are academically able, as well as those who are less prepared for university study.

In its advice on how to interpret the threshold standards, TEQSA could require higher education providers to make information on maximum completion times more easily available. It could give other advice relevant to the student's decision to study part-time, including information about the risks involved in studying part-time, and how these can be reduced. TEQSA could also require providers to check that students enrolling on a part-time basis have credible plans for completing their degrees, as discussed in \Cref{sec:4.3}.

\section{Improvements in monitoring}\label{sec:7.2}

TEQSA does not micro-manage universities. They are all self-accrediting, which means they approve their own courses, including their admission requirements. Instead of examining the details of thousands of university courses, TEQSA annually uses an institution-level risk-based approach, only examining further when there is some reason to do so. It uses various indicators as triggers for consideration, but makes decisions based on expert judgment.\footcite{TEQSA2016e} \CenturyFootnote
These indicators include first-year attrition rates, pass/fail rates for all subjects, and numbers of course completions.

TEQSA should continue with this approach, but could adapt its annual monitoring to identify problems that may currently be hidden. A weakness of institution-level monitoring is that satisfactory overall outcomes can hide poor results for particular groups of students. For example, although part-time and low-ATAR students are an increasing share of all commencing bachelor-degree students, institution-level results are dominated by the much larger groups of full-time and high-ATAR students. Given that part-time and low-ATAR students are at elevated risk of not completing their courses, their attrition rates should be examined separately. The relevant data is already collected and available to TEQSA; it would not require a lot of additional work to analyse it.\footnote{In some cases, this is already done when TEQSA has reasons for concern about particular groups of students. See \textcite[][]{TEQSA2017f}.}

Institution pass/fail rates are currently calculated as percentages of all subjects taken. While this indicator is interesting, students failing all or many subjects are the main attrition concern (\Cref{subsec:3.1.1}). A significant proportion of students failing multiple subjects could be a sign of poor admission practices, not managing disengaged students well, or problems with curriculum or teaching. As with attrition rates, the data needed to monitor fail rates is already collected and available to TEQSA\@. Where potential issues become apparent, TEQSA should investigate in more detail to find the problem's source.

If TEQSA changes its annual monitoring practices, the universities not already carefully tracking their outcomes will have an added incentive to do so. This would not be a major additional impost, because the threshold standards already require monitoring on a range of levels including course, cohort, educational background and entry pathway.\footcite[][section~1.3]{DepartmentofEducationandTraining2015n}

