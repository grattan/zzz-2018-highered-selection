\chapter{Mutual selection analysis methodology}\label{chap:a}

The mutual selection analysis focuses on domestic bachelor degree applicants for the 2014 academic year. It uses both the higher education student enrolment data and the applications and offers data collected by the Department of Education and Training. The applications and offers dataset includes both applications through tertiary admission centres and direct applications to universities.

Students were identified and tracked across datasets using the Commonwealth Higher Education Student Support Number (CHESSN). About 1 per cent of domestic bachelor degree applications do not have a CHESSN and were excluded. Since offers from the Universities Admissions Centre (the admission centre for NSW and ACT) and the University of Tasmania have a high proportion of unknown offer responses, they were also excluded.\footnote{UAC's unknown offer response rate is 23 per cent, UTAS's is 33 per cent, \textcite{DepartmentofEducationandTraininga}.}

Students can apply to multiple universities and in different states (through tertiary admission centres), so some students receive more than one offer. For students with multiple offers, their best response is chosen according to the hierarchy: accept (including defer), reject (including offers deemed to have lapsed or superseded), and response unknown. For example, if an applicant receives two offers and defers one and lets the other lapse, this analysis assumes she accepts.

In 2014, the Victorian Tertiary Admission Centre had a significant proportion of supplementary offers or unsolicited offers.\footnote{Offers without a corresponding application for the course.} 
These offers are ignored unless they were accepted or deferred.\footnote{About 1 per cent of supplementary offers were accepted or deferred.}

For students who accept or defer an offer, they are considered `enrolled at the census date in semester one' if they enrol in at least one bachelor-degree subject in their first semester. A student's first semester is assumed to be whenever she first enrols.\footnote{Starting from the 2014 academic year.} 
For example, if a student accepted an offer for the 2014 academic year and undertook no bachelor-level subject in semester one of the 2014 academic year and some subjects in semester two, her first enrolment semester would be semester two of the 2014 academic year. Her second semester would correspond to semester one of the 2015 academic year, and her second year would be semester 2 of the 2015 academic year and semester 1 of the 2016 academic year. This time-shift approach for initial enrolment allows deferred students to be included while ensuring that their enrolment patterns are consistently analysed compared with those who commence in semester one.\footnote{Deferred students include any students who deferred or accepted an offer and did not enrol in at least semester one of the 2014 academic year.}

Students who are not enrolled by semester two of the 2015 academic year are classified as `not enrolled at the census date semester one' even if they have accepted or deferred an offer.

Once students commence, the analysis proceeds from their first semester. For example, students who first enrolled in semester one of the 2014 academic year and did not take any bachelor-degree level subjects in semester two of the 2014 academic year but then took some subjects in semester one of the 2015 academic year would be considered `not enrolled in semester two' but `enrolled in second year'. These students are denoted in red on left panel (participating) on \Vref{fig:2} and represent roughly 2 per cent of total applicants.

% \section{Attrition rates}\label{attrition-rates}

    % The attrition rates generated through this mutual selection analysis are greater than those reported by the Department of Education.\footnote{The department reports these figures annually. \emph{Selected Higher Education Statistics} -- \emph{Full Year Data (2017).}} Our analysis focuses strictly on bachelor degree participation and completion. The Department considers only commencing bachelor students but counts participation and completion in any subsequent award course. For example, a student who leaves their bachelor degree and begins a diploma will be counted as `participating' by the department, but as `not participating' in Grattan's analysis.
    
    % In addition, our mutual selection analysis uses a time-shift approach (see section 9.1 above). The Department's academic calendar year approach counts as `participating in second year' a student who commences in the year's second semester, continues into the next semester, and then drops out. This approach reduces the first-year attrition rates of second-semester commencers.
    
    % These differences make a significant impact on the headline attrition figure. In our analysis shown in Figure 2 on page 8 the domestic bachelor first-year attrition rate was 19 per cent in 2014. The Department reports a first-year attrition rate of 15 per cent for the same group.

