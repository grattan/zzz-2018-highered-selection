\chapter{New census date policies to increase student protection }\label{chap:8}

With better advice, some prospective university students would not apply or enrol. But inevitably substantial numbers of people will still enrol only to find that their course is not right for them. More can be done to protect these students from unnecessary costs.

There are a number of options. The Government could give the census date a name that highlights its financial consequences for students. Universities could send students text messages shortly before the census date, reminding them of the deadline to dis-enrol and avoid costs, along with providing contacts for support. Universities could follow up more intensively than they do at present with students showing a pattern of disengagement before the census date of their second semester. And universities could require students to proactively opt-in to confirm their enrolment by the census date.

Universities should trial these options to identify which strikes the best balance between protecting disengaged students from wasting time and money, and imposing additional administrative burdens on all students.

\section{Give the census date a more meaningful name}\label{sec:8.0}

Many students do not understand that the census date is the day when they become liable to pay student contributions or fees. A small Grattan Institute face-to-face survey found high levels of ignorance.\footnote{Grattan Institute conducted a face-to-face survey of about 50 students from each of the University of Melbourne, RMIT University, Swinburne University and the Australian Catholic University's Fitzroy campus in Melbourne on 23 April 2018.} Less than 40 per cent of the students sampled understood the census date. The others did not know what it was, or confused it with some other important date, such as the last day to change subjects or the last day to withdraw from a subject without fail. A small majority of survey respondents were in their second or later year, suggesting that this is not just an issue with first-year students.

A simple name change could help students appreciate the date's significance. The term `census date' reflects the perspective of university and government administrators -- they need a count of students. But from the point-of-view of students what matters is not how many people have enrolled, but the fact that they are now liable to pay for their subjects. The name of the date should reflect this. The exact terminology should be chosen after appropriate testing with students. But it needs to highlight that this is a day with financial consequences.

\section{Reminders about census dates}\label{sec:8.1}

Even students who understand the census date do not always know when it is or forget about it. In the census date survey, more than one-in-ten students missed the census date for a subject they wanted to drop.

Universities do typically remind students of the census date, but usually by email. As email inboxes are often cluttered, and email is not necessarily the primary form of electronic communication for students, it is easy for this message to be missed. Other methods of communication include information on learning management systems, and notices on campus. These are worthwhile, but won't be seen by students who have already disengaged.

Though emails are often overlooked, mobile phone text messages are known to be effective as reminders.\footcite{McLean2016} 
Mobile phone use is near universal in the main student age groups.\footcite{Deloitte2017} 
Text messages to all students a week or so before each census date would help those who want to drop a subject, as well as those who want to leave their course. The message could also inform or remind students of where they can find advice or help with any problems they are having. Some universities already use text messages to get important information to students.

But text messages will not necessarily protect very disengaged students, who may ignore messages from their university, or not take the necessary steps to discontinue their course.

\section{Requiring universities to ensure students are engaged by the census date }\label{sec:8.2}

Eventually all students in bachelor degrees reach a decision point where the default decision is to leave. This is re-enrolment, which is usually annual. Students who don't re-enrol don't incur any more HELP debt. \Cref{sec:1.1} reported that of the students who were still enrolled in second semester of their first year, 9 per cent were not enrolled at any time in their second year. The source of this data cannot distinguish between students who formally left (discontinued, took a leave of absence, or were excluded by their university) and those who just did not re-enrol.\footnote{See \textcite[][20--21]{Harvey2017a}, on the general issue. Federation University data shows that more than half of its first year attrition was from people who did not re-enrol and did not inform the university: \textcite[][]{FederationUniversity2018}.}

While very disengaged students will eventually exit the system, they can accrue a year of HELP debt before they do so. Modified census date rules, that require universities to take action when students are disengaged, could minimise the number of students needlessly accruing HELP debt.

In the non-university higher education sector, this is already required. Legislation passed in 2017 puts additional responsibilities on these providers to check that students are `genuine' in each subject by the census date. Tests of genuineness include participating in assessment activities, for online students logging in to the course website, and being `reasonably engaged' in the course.\footnote{At the time of writing, the detailed rules have not been published. However, the Government has said that they will be similar to those in the \emph{VET Student Loan Rules 2016}, from which these examples are taken.} 
If students are deemed not genuine, they lose their FEE-HELP loan eligibility.\footnote{FEE-HELP is the income contingent loan scheme for full-fee students. The legislation is the \emph{Education Legislation Amendment (Provider Integrity and Other Measures) Act 2017.}} Unless they can afford to pay upfront, students in this situation would have to leave their course.

Encouraging students to leave before the census date if they have not engaged with their studies, as some universities already do (\Cref{sec:6.2}), is good practice. But the additional step of effectively disenrolling a student needs more protections than provided by the `genuine student' rules applying to FEE-HELP students. A student with some engagement, albeit not enough to pass unless increased, could be deemed not genuine. If the rules for full-fee students were applied to Commonwealth-supported students, if a university decides that a student is not genuine, the student would effectively be expelled.\footnote{The FEE-HELP students covered by the current legislation can remain enrolled, but have to pay up-front if found not to be `genuine'. For a Commonwealth-supported student, ineligibility for funding would mean cancellation of enrolment: Higher Education Support Act 2003, section~36-40. There may be some legal ways to work around this, such as the student declining to be Commonwealth supported under section~36-10(3), or continuing in the same subjects on a not-for-degree basis. However, these would involve upfront fees.}

In some cases, higher education providers should disenrol very disengaged students. But this major decision needs clear rules, announced in advance, about what students need to do to remain enrolled, and about what universities should do to encourage students to engage or to disenrol. Even nil or negligible engagement might sometimes have a reasonable explanation, such as illness. Students should be able to appeal against an exclusion decision, as indeed higher education regulation requires.\footnote{Higher Education Support Act 2003, section~19-45; \textcite[][section~2.4]{DepartmentofEducationandTraining2015n}.}

The first census date timelines will often be too tight to follow all these steps properly. These procedural and practical difficulties are reflected in university practice. Griffith University recognises the need to distinguish `slow starters', who show some engagement but are not attending small classes or submitting work, from `no shows', who never turned up. Despite having the power to disenrol a student during semester, in practice the University of Tasmania withdraws the student from the unit only, without registering a `fail'.\footcite{Brown2017} The student stays enrolled and incurs a HELP debt, but their academic record is protected.

There are other practical difficulties with excluding students by the first census date. Despite the spread of learning analytics software, not all institutions can easily monitor their students and follow up with those showing signs of disengagement. The University of Tasmania's experience shows that managing disengaged students early is not just a technology issue. It changed course content, to ensure there were sufficient assessment tasks before the census date, to aid informed decision making. The University of Tasmania created centralised teams to follow up with disengaged students, because the task can be too much for course coordinators whose principal responsibility is to the students who are engaged with their studies.

By the second semester census date, it is easier to manage many of the informational and organisational issues. Each student's patterns of engagement will be easier to determine. By then, the results of course assessment tasks, which are already recorded by all institutions, would be available. A consistent pattern of failing to submit assignments, not sitting exams, and very low marks would be strong evidence of disengagement. At most universities, there are many weeks between the end of first semester and the second semester census date. This allows time for careful assessment of each student's case, and time for students to dispute decisions by the university.

\section{Performance funding}\label{sec:8.3}

The Government has announced that from 2020 it will link funding to university performance, with attrition likely to be a performance indicator.\footnote{Australian Government (2017a).} The details are yet to be announced, but it is likely that universities with high rates of students leaving before their second year will receive a lower, or no, funding increase.

Performance funding would encourage universities to focus on disengaged students before the first census date of each new enrolment, as discussed in \Cref{sec:6.2}. Students who leave before the first census date are never counted as enrolled for official attrition rate purposes (\Chapref{chap:1}). After the first census date, a financial reward for second-year enrolment would encourage universities to support students who are considering leaving, but who could still successfully complete their course (\Cref{sec:6.3}).

But `performance' in attrition is complex. For most students, staying enrolled and completing a degree is the right outcome, and performance funding supports that. But for students who are not likely to complete, their best interests are better served by concluding their enrolment at least cost in time and money. Performance funding encourages that in the early weeks of first year, but less so subsequently. It provides no incentive to review student progress after first semester.

Regardless of the incentives, in Australian higher education performance funding has a credibility problem. Previous performance schemes suffered from changing goals and financial rewards that were abolished to provide Budget savings.\footnote{See \textcite[][chapter~5]{Coaldrake2016} for a discussion of previous performance funding schemes in Australia.} The Government's 2020 proposal lacks detail and has not been legislated. With uncertain benefits, universities may not change their existing policies on student retention.

\section{An opt-in course confirmation}\label{sec:8.4}

With census date reminders, the default decision is to remain enrolled. Requiring universities to monitor disengaged students more carefully would be more likely to protect them from HELP debt at the second rather than the first census date. But the first census date is when the number of enrolled but potentially disengaged students is likely to be near its peak.

One possible policy response is to have a date early in first semester that is more like re-enrolment: the default decision is to leave rather than to stay. The date would be opt-in rather than opt-out: only students who agree to stay enrolled, and who formally accept enrolment's financial consequences, would pay student contributions or incur a HELP debt.\footnote{International students would be excluded from any legal requirement for an opt-in system. Their system of applying for entry is already designed to screen out experimental, keeping-options-open enrolment. It includes non-refundable fees, applying for a visa, and travel costs. They could also lose their right to remain in Australia if they leave their course. Opt-in has some potential to benefit domestic postgraduate students. However, they are a more informed group who should already know about the census date if they completed their undergraduate education in Australia. Their patterns of subject fails and course attrition should be researched further before considering opt-in.}

Like re-enrolment, opting in would be a course-level decision, which makes it different from the census date, which can be a more limited decision about subjects taken. But whenever possible, to simplify the process, an opt-in course confirmation date should coincide with the subject-level census date applying to first-year students.\footnote{Universities could have different subject census dates and course confirmations, but the later student decision would over-ride the first. This might occur for subjects typically taken by commencing students that can also be taken by students further into their course.} Its aim is a quick and cheap exit for disengaged students, while minimising inconvenience to others.

To opt in, students would log back into their university's enrolment system, be shown which subjects they are enrolled in and how much these cost, and be asked to confirm, modify or end their enrolment. Students who choose to end their enrolment, or who don't respond to the request, would not incur any future charges.\footnote{In the VET Student Loans system, students taking out loans are asked to confirm their enrolment with the Department of Education and Training every four months; \textcite[][]{AustralianGovernment2017a}. This regular opt-in reflects a history of serious problems in the VET sector. The issues in the higher education sector are much less serious, and one opt-in per course should be enough.} Universities could offer deferrals to students who want to postpone rather than terminate their studies.

To help students make an informed choice, a confirmation decision date should give them time to experience their course. In most cases existing census dates achieve this goal. As noted in \Cref{sec:6.2}, the median census date is four weeks into the teaching period, and few universities have standard census dates less than three weeks into the teaching period. The substantial numbers of students who leave in this period suggests that three or four weeks is enough time for many students to form a view about whether university is for them.

When universities depart from semester or trimester systems, current census date rules, based on a percentage of the way between commencement and completion of a subject, can produce much shorter time periods. For example, under Victoria University's new first-year model, students take subjects sequentially for intensive four-week periods rather than simultaneously over a conventional semester of around 16 weeks. The census dates for these Victoria University subjects are currently about a week into the teaching period. Although these quick subjects may provide a wider range of academic experiences early on, a week is still a very short time in which to experience the course and university life.

To give students time to decide, the final date for a course confirmation should be set in days. A minimum 21 days for students to confirm or leave would strike a balance between giving them time to decide and protecting the financial interests of universities. The longer students have in a try-before-you buy phase, the more resources universities invest in them.

For most universities, a 21-day confirmation period would make no difference to the time students have to try before they buy. They are likely to keep their census dates, which are often established parts of university academic calendars. Under an opt-in system, universities may also want more than three weeks to evaluate student progress and follow-up on students at risk of not completing. Some universities might use a later date as a selling point to prospective students.

A new opt-in system would need provisions to protect against mistaken exits. There should not be a single high-stakes day on which students can accidentally disenrol themselves.

To minimise accidental disenrolment, there could be a course confirmation period prior to the final decision day. For example, the confirmation period could start 14 days into the teaching period, to ensure that students have some experience of university, and end on the course confirmation date chosen by the university or after 7 days, whichever is the later. A confirmation period gives universities time to advise and warn students of the impending date. It gives students multiple days on which they can notify the university of their decision, before it becomes final at the end of the confirmation period. There could also be a time after the confirmation day when students could, for a late fee, correct their own or the default decision to discontinue.

Universities would have reason to promote course confirmation in a way they don't currently emphasise the census date, since they could lose substantial revenue from erroneous disenrolment. Course confirmation would create an added incentive for universities to identify and support disengaged students. They probably now lose students they could retain, as well as temporarily keep students who are too disengaged even to end their enrolment. The Grattan survey of people who left university without completing found that, knowing that they know now, nearly 20 per cent would not drop out.\footnote{Of those who never acquired a degree.} 
Another survey of students who had left one university found that around 40 per cent believed that the university could have done something to encourage them to stay.\footcite[][35]{Harvey2017a}

\section{Trialling different options}\label{sec:8.5}

Australia's higher education system is very open to people who want to try to get a degree. But in providing opportunities, it also generates risks that some people will get little but a HELP debt from their enrolment. The main policy measure to limit these risks is a payment date that, compared to other countries or international students in Australia, is well into the teaching period.

This chapter has suggested ways to better protect students whose higher education experiment is not working out. Each suggestion has strengths and weaknesses, and each could generate a variety of behavioural responses from students and universities.

One way to proceed may be to run policy trials to measure the effectiveness of different measures. Text message reminders are very suitable for a randomised controlled trial. Universities with good learning analytics could investigate how students with different levels of engagement respond to reminders about the census date. If text message reminders prompt significant numbers of disengaged students to seek help or discontinue, more complicated measures such as requiring students to opt-in by the first census date may not be necessary.

Universities under pressure from TEQSA to improve their attrition rates are free to impose conditions on enrolment that let them disenrol very disengaged students. They could also create an opt-in course confirmation date. Policy makers could use these experiments to inform policy for the sector as a whole.