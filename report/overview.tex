\begin{overview}


Nearly a quarter of a million students will start a bachelor degree in Australia in 2018, but more than 50,000 of them will leave university without getting a degree.

Dropping out is not always a bad outcome. Surveys of school and first-year university students show that many are uncertain about their direction. Enrolling can help students decide what they want to do. If students discover that university is not for them, and leave quickly, it costs them little in time and money.

Partially complete degrees can have other advantages. Many people who did not finish their course found it interesting, learned useful skills, and made lasting friendships and connections. Often they say that if they had their time over again, they would still begin their course, suggesting they believe that their enrolment brought more benefits than costs.

Yet for a significant minority, an incomplete degree leaves them with debt and regret. Nearly two-thirds believe they would have been better off if they had finished. Nearly 40 per cent of students who dropped out would not begin their degree again knowing what they know now, and about a third of them believe they received no benefits from their course. These students do not get value for their time and money.

Much of the risk of dropping out is foreseeable. Part-time students are the most likely to drop out. Many try to combine study with paid work and family, but discover they can’t manage their competing commitments. Students who enrol in three or four subjects a year -- half as many as a full-time student -- have only about a 50 per cent chance of completing their course in eight years. Students who enrol full-time have about an 80 per cent chance. 

School results are important. Students with ATARs below 60 are twice as likely to drop out of university than otherwise similar students with ATARs above 90.

With better advice, some prospective part-time students may opt to study full-time. And some low-ATAR students would take a vocational education course instead. Some may not study at all, but look for a job instead. 

Governments and universities should do more to alert prospective students to their risk factors. The Commonwealth Government’s Quality Indicators for Learning and Teaching information website should include a guide to students’ completion prospects. Universities should check that part-time students have realistic course completion plans. 

Some students who drop out never seriously engage with their course, and needlessly accrue HELP debt before they leave. This report recommends new ways to protect students from unnecessary financial burdens.

Australia’s higher education system lets people try out university. It recognises that not enrolling, as well as dropping out, has costs and risks. But Australia should do more to reduce the number of drop-outs. With some changes, Australia’s higher education system could make dropping out less common and less costly. 


\end{overview}