 \begin{recommendations}
 
\subsubsection{For people thinking about applying for university}

    The Government’s student website, QILT, should include personalised information about the risk of not completing a degree. It should also advise on how to reduce this risk. 
    
    University web pages for future students should clearly state what part-time students need to do to finish the course in the maximum time allowed. 

\subsubsection{When universities are enrolling or re-enrolling students}

    Universities should check that students take enough subjects to complete their degree in the maximum time, or that the student has a credible plan for catching up. 

\subsubsection{Before the census date when students become liable to pay for their subjects}

    All students should receive more effective communication about the importance and timing of their census date, so they don’t pay for subjects that they are unlikely to complete. 
    
    If students are disengaged before the census date, and don’t commit to re-engaging, universities should cancel their enrolment. 
    
    If disengaged commencing students remain an issue after other methods of protecting them are tried, the Government could require students to confirm their enrolment, or opt-in, a few weeks into term. Students who did not confirm would no longer be enrolled and would not incur a HELP debt. 


\subsubsection{For the Tertiary Education Quality and Standards Agency}

    The regulator, TEQSA, should pay more attention to what universities tell prospective part-time students about how many subjects they need to take, and whether universities are enrolling part-time students who do not have credible plans for completing their degree. 
    
    TEQSA’s annual monitoring of higher education providers should include examining outcomes for students at high risk of not completing their degree, such as part-time students and students failing many subjects. 


\end{recommendations}